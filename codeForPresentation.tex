\documentclass{beamer}
\usetheme{AnnArbor}
\usepackage{pgfplots, pgfplotstable}
\usepackage{amsmath}

\title{PROBABILITY and STATISTICS \\ BATCH B1 GROUP G2}
\author{RAVI - IIT2018108 \\ MEGHNA - IIT2018109 \\ VIKASH - IIT2018110 \\ JISHAN -  IIT2018111 \\ AKHIL - IIT2018112}
\institute{Indian Institute of Information Technology,Allahabad}
\date{\today}

\begin{document}
\begin{frame}
\titlepage
\end{frame}

\begin{frame}{Question 1}
\large {A certain population consists of 20\% children, 30\% adolescents, and 50\% adults. The probabilities that a member of this population catches the flu are 0.45 for a child, 0.2 for an adolescent, and 0.15 for an adult.\\
(a) What is the probability that a randomly selected member of this population has
the flu?\\
(b) What is the probability that a randomly selected person with the flu is an adult?}

\end{frame}
\begin{frame}{Solution}
(a) Given: $P$(children)= 0.2 , $P$(adolescents)= 0.3 , $P$(adult)= 0.5 \\
Also: $P$(flu \mid children)= 0.45 \\ \hspace{9mm} $P$(flu \mid adolescents)= 0.2 \\ \hspace{9mm} $P$(flu \mid adult)= 0.15 

\vspace{2mm}\hspace{10mm}
By Total Probability Theorem : \\
\vspace{2mm}\hspace{10mm} $P$(E) = $\sum_{k=1}^{n} $P$(A_k)$P$(A\mid A_k)$ \\
\vspace{2mm}\hspace{20mm}        = $P$(children)\times $P$(flu \mid children) +
\\ \vspace{2mm}\hspace{20mm} $P$(adolescents)\times $P$(flu \mid adolescents) +
\\ \vspace{2mm}\hspace{20mm} $P$(adult)\times $P$(flu \mid adult) \\
\vspace{2mm}\hspace{10mm} = 0.2\times0.45 + 0.3\times0.2 + 0.5\times0.15\\
\vspace{2mm}\hspace{10mm} = 0.09 + 0.06 + 0.075 \\
\vspace{2mm}\hspace{10mm} = 0.225 \\
\end{frame}

\begin{frame}{Solution}
\large{
(b) So, by using Baye's theorem, we get:
\vspace{2mm}\hspace{10mm}\\
$P$(adult \mid flu) = \dfrac{$P$(flu \mid adult) \times $P$(adult)}{$P$(flu)}\\
\vspace{2mm}\hspace{10mm}
 = \dfrac{0.15\times0.5}{0.225}             \hspace{50mm}(from : part-a)\\
\vspace{2mm}\hspace{10mm}
So, $P$(adult \mid flu) = \frac{1}{3}\\}
\end{frame}

\begin{frame}{Question 2}
\large {
Let X be a discrete random variable with support S_X = \{0, 1, 2, 3, 4\}, P(\{X = 0\}) =
P(\{X = 1\}) = \dfrac{1}{10} \\
P(\{X = 2\}) = P(\{X = 3\}) = P(\{X = 4\}) = \dfrac{4}{15}.\\
Find the distribution
function of X and sketch its graph.}
\end{frame}

\begin{frame}{Solution}
Given:- S_X = \{0, 1, 2, 3, 4\} , P(\{X = 0\}) =
P(\{X = 1\}) = \dfrac{1}{10} \\
P(\{X = 2\}) = P(\{X = 3\}) = P(\{X = 4\}) = \dfrac{4}{15}.\\
\vspace{2mm}
Now, distribution function is given by :\\
\begin{equation*}
    F_X(x) = \left\{\begin{array}{lr}
        0 & \text{for } x < 0\\
        \vspace{2mm}
        \frac{1}{10} & \text{for } 0\leq x<1\\
        \vspace{2mm}
        \frac{1}{10} + \frac{1}{10} = \frac{1}{5} & \text{for } 1\leq x<2\\
        \vspace{2mm}
        \frac{1}{5} + \frac{4}{15} = \frac{7}{15}  & \text{for } 2\leq x<3\\
        \vspace{2mm}
        \frac{7}{15} + \frac{4}{15} = \frac{11}{15}  & \text{for } 3\leq x<4\\
        \vspace{2mm}
        \frac{11}{15} + \frac{4}{15} = 1  & \text{for } x \geq 4
        \end{array}\right\}
\end{equation*}
\end{frame}

\makeatletter
\long\def\ifnodedefined#1#2#3{%
    \@ifundefined{pgf@sh@ns@#1}{#3}{#2}%
}

\pgfplotsset{
    discontinuous/.style={
    scatter,
    scatter/@pre marker code/.code={
        \ifnodedefined{marker}{
            \pgfpointdiff{\pgfpointanchor{marker}{center}}%
             {\pgfpoint{0}{0}}%
             \ifdim\pgf@y>0pt
                \tikzset{options/.style={mark=*, fill=white}}
                \draw [densely dashed] (marker-|0,0) -- (0,0);
                \draw plot [mark=*] coordinates {(marker-|0,0)};
             \else
                \tikzset{options/.style={mark=none}}
             \fi
        }{
            \tikzset{options/.style={mark=none}}        
        }
        \coordinate (marker) at (0,0);
        \begin{scope}[options]
    },
    scatter/@post marker code/.code={\end{scope}}
    }
}

\makeatother

\begin{document}
Graph of distribution function is:\\
{\centering
\begin{tikzpicture}
\begin{axis}[
    clip=false,
    jump mark left,
    ymin=0,ymax=1,
    xmin=0, xmax=5,
    every axis plot/.style={very thick},
    discontinuous,
    table/create on use/cumulative distribution/.style={
        create col/expr={\pgfmathaccuma + \thisrow{f(x)}}   
    }
]
\addplot [black] table [y=cumulative distribution]{
x f(x)
0 1/10
1 1/10
2 4/15
3 4/15
4 4/15
5 0
};
\end{axis}
\end{tikzpicture}
\par}
\end{document}


\begin{frame}

\hspace{25mm}
\Huge Thank You
\end{frame}


\end{document}